\documentclass[]{article}
\usepackage{babel}
\usepackage{verbatim}

\begin{document}

\title{Projet d'Algorithmique}
\author{David Galichet et Baptiste Fontaine}
\date{28 novembre 2013}
\maketitle

\chapter{Implémentation}

Le projet a été écrit avec Python 3 sans utilisation de bibliothèques externes.

\section{Mise en route}
\subsection{Installation}

Le code est situé dans \texttt{src}, et ne nécessite pas d'installation
particulière. Assurez-vous cependant d'avoir au moins Python 3.1.

\subsection{Utilisation}

Le programme s'appelle \texttt{wrapper}, il suffit de le lancer avec l'option
\texttt{--help} pour avoir un aperçu de son utilisation ainsi qu'une liste des
algorithmes disponibles. Il lit son entrée sur l'entrée standard, et écrit le
résultat sur la sortie standard. Il est possible de choisir l'algorithme à
utiliser en utilisant l'option \texttt{--algo}, ainsi que de changer la largeur
de la page (79 par défaut) avec \texttt{-w}.

\begin{verbatim}
# formatter "in.txt" avec l'algorithme "naive-greedy"
# et une largeur de 76
./wrapper --algo naive-greedy -w 76 < in.txt
\end{verbatim}

Il est également possible d'obtenir plus d'informations sur un algorithm en
utilisant l'option \texttt{--info}.

\begin{verbatim}
./wrapper --info naive-greedy
\end{verbatim}

\section{Fonctionnement}

\chapter{Algorithmes}

\section{Diviser pour régner}

\section{Gloutons}

\subsection{Glouton naïf}

\section{Programmation dynamique}

\end{document}
